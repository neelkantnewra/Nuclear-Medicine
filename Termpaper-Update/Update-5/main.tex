
\documentclass{article}
\usepackage[utf8]{inputenc}
\usepackage{graphicx}
\usepackage[paper=a4paper,left=3cm, right=3cm, top=2cm]{geometry}
\usepackage{blindtext}
\usepackage{multicol}
\usepackage{pgfplots}
\usepackage{float}
\usepackage{biblatex} %Imports biblatex package
\usepackage{amsmath}
\pgfplotsset{compat = newest}

\title{\textbf{Recent Development in inorganic scintillator-based optic fiber dosimeters}}
\author{Neelkant Newra \\ \textit{Department of Biomedical Engineering} \\ National Institute of Technology, Raipur \\ Email: newra008@gmail.com}

\begin{document}

\maketitle


\section*{Aim}
To study the various development in the area of Scintillator-based optic fiber dosimeter, concentrating especially on the inorganic Scintillator. 
\\

\begin{multicols}{2}
\noindent There are two types of dosimeter available:
\begin{enumerate}
    \item Diode based i.e MOSFET
    \item Scintillator based
\end{enumerate}

\section{Diode Based}

\subsection{Advantages}
\begin{itemize}
    \item good sensitivity
    \item small size
    \item real-time readout
\end{itemize}

\subsection{Disadvantage}
\begin{itemize}
    \item Expensive
    \item incident angle dependent
    \item limited durability
\end{itemize}

\section{Scintillator based}

\subsection{Advantages}
\begin{itemize}
    
    \item passive detection
    \item small size
    \item linear response to dose rate
    \item energy independence 
    \item immunity to electromagnetic interference 
    \item good medical robustness
    \item capacity for multiplexing 
    
\end{itemize}

\noindent Scintillator based are further classified on the basis of what material they are using for fabrication
\begin{enumerate}
    \item \textbf{organic scintillator:} This are combination of plastic + organic scintillator comprising of aromatic hydrocarbon molecules.
    \item \textbf{inorganic scintillator:} although application of inorganic scintillator are very less in scintillator based dosimeter due to their non-linear response to x-ray energies under 100 keV and non watery equivalency, yet current day research interest is raised due to its "high sensitivity to low  radiation dose rate.
\end{enumerate}
\end{multicols}




\end{document}
